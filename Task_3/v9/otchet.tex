\documentclass[a4paper]{article} 
\usepackage[12pt]{extsizes} 
\usepackage[utf8]{inputenc} 
\usepackage[russian]{babel} 
\usepackage{setspace,amsmath} 
\usepackage[left=20mm, top=20mm, right=20mm, bottom=20mm, nohead, footskip=10mm, landscape]{geometry} 
\usepackage{indentfirst} 
\usepackage{multirow} 
\usepackage{fourier} 
\usepackage{array} 
\usepackage{makecell} 
 
\newcolumntype{P}[1]{>{\centering\arraybackslash}p{#1}} 
 
\newcolumntype{M}[1]{>{\centering\arraybackslash}m{#1}} 
 
\begin{document} 
\renewcommand{\arraystretch}{2} 
\begin{center} 
{\Large Решение краевой задачи ОДУ сеточным методом с Использовнием метода прогонки.} 
\bigskip 
\end{center} 
Выполнил: Капелюшников Андрей Сергеевич 
 
\begin{equation*} 
\begin{cases} 
u'' (x) + u' (x) - \frac{1}{x}u=\frac{1+x}{x} 
\\ 
0,5 < x <1 
\\ 
u(0,5)=\frac{-1}{2 \ln2} 
\\ 
u'(1)=0 
\end{cases} 
\end{equation*} 
\begin{table}[h!] 
\begin{tabular}{ccccccccc} 
\hline 
$n_1/n_2$  & 25 / 50  & 50 / 100  & 100 / 200  & 200 / 500  & 500 / 1000  & 1000 / 2000  & 2000 / 4000  & 4000 / 8000 \\ 
\hline 
 $||\cdot||_{\infty}$  & -0 & -0 & -1.49167e-154 & -0 & -0 & 0 & 0 & 0\\ 
\hline 
 $||\cdot||_{2}$  & 3.95253e-321 & 7.90505e-321 & 1.58101e-320 & 3.95253e-320 & 7.90505e-320 & 1.58101e-319 & 3.16202e-319 & 6.32404e-319\\ 
\hline 
 $||\cdot||_{1}$  & 3.95253e-321 & 7.90505e-321 & 1.58101e-320 & 3.95253e-320 & 7.90505e-320 & 1.58101e-319 & 3.16202e-319 & 6.32404e-319\\ 
\hline 
\end{tabular} 
\end{table} 
\end{document} 
